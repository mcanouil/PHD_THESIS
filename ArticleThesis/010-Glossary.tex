\newacronym{dt2}{DT2}{Diabète de Type 2}
\newacronym{ada}{ADA}{Association Américaine pour le Diabète}
\newacronym{hba1c}{HbA1c}{Hémoglobine glyquée A1c}
\newacronym{imc}{IMC}{Indice de Masse Corporelle}
\newacronym{nafld}{NAFLD}{Stéatose hépatique non-alcoolique}
\newacronym{nash}{NASH}{Stéato-hépatite non-alcoolique}
\newacronym{pop}{POP}{Polluant Organique Persistant}
\newacronym{rygb}{RYGB}{Bypass en roux-en-Y}

\newglossaryentry{BODY MASS INDEX}{
    name={BODY MASS INDEX},
    description={Body Mass Index (BMI), also known as the Quetelet Index, is a person's weight in
    kilograms divided by the square of height in meters (kg/m2). A table of BMI values from
    height and weight is available at https://en.wikipedia.org/wiki/Body_mass_index}
}
\newglossaryentry{COMMON GENOMIC VARIANT}{
    name={COMMON GENOMIC VARIANT},
    description={Single nucleotide variation in genetic sequences where the less prevalent form (minor
    allele) occurs at a frequency of 1% or greater in the human population under
    investigation.}
}
\newglossaryentry{EFFECT SIZE ESTIMATE}{
name={EFFECT SIZE ESTIMATE},
    description={A measure of the magnitude of the difference in allele frequencies between two groups
    or between group phenotype values. The estimate is typically expressed as an odds
    ratio for a case:control GWA study or as a regression coefficient for continuous traits but
    there are many other ways to quantify an effect size.}
}
\newglossaryentry{ENCODE CONSORTIUM}{
    name={ENCODE CONSORTIUM},
    description={Encyclopedia of DNA Elements. A public research consortium launched in 2003 by the
    National Human Genome Research Institute (NHGRI) with the aim of identifying and
    cataloging all functional elements in the human genome.}
}
\newglossaryentry{EXPRESSION QUANTITATIVE TRAIT LOCI}{
    name={EXPRESSION QUANTITATIVE TRAIT LOCI},
    description={Regions of the genome containing DNA sequence variants that influence the expression
    level of one or more genes.}
}
\newglossaryentry{GENE-BEHAVIOR INTERACTION}{
    name={GENE-BEHAVIOR INTERACTION},
    description={A gene-behavior interaction is present when the response to a behavior pattern or a
    behavioral change is conditional on the genotype. For instance, a diet rich in polyunsaturated
    fat may have variable effects on adiposity depending on the genotype of
    the person at a few loci. It is also referred to as a gene-environmental interaction.}
}
\newglossaryentry{GENOME-WIDE SIGNIFICANT}{
    name={GENOME-WIDE SIGNIFICANT},
    description={It typically applies to an association p-value for a single nucleotide polymorphisms in a
    genome-wide association study. A SNP with an association p-value<0.05, after
    correction for the number of SNPs tested (Bonferroni correction), is considered to be
    genome-wide significant. For 1 million SNPs tested, this equates to a SNP with nominal
    p-value of 5X10E-08.}
}
\newglossaryentry{GTEx}{
    name={GTEx},
    description={Refers to the Genotype Tissue Expression project. Launched by the NIH in 2010, it
    aims to provide a pubic data resource enabling the study of gene expression and
    regulation and its relationship to genetic variation in humans.}
}
\newglossaryentry{GWA}{
    name={GWA},
    description={Genome-wide association is an approach involving the simultaneous scanning of
    millions of markers (single nucleotide polymorphisms) across the entire genome with
    the goal of discovering genetic variants that are associated with a particular disease or
    trait.}
}
\newglossaryentry{HERITABILITY}{
    name={HERITABILITY},
    description={An estimate of the contribution of genetic variation to a phenotype among individuals in
    a given population.}
}
\newglossaryentry{METABOLIC RATE}{
    name={METABOLIC RATE},
    description={The rate of metabolic energy expenditure to meet the energy needs of the body. For
    instance, resting metabolic rate is the rate of caloric expenditure to maintain the basic
    biological functions of the body at rest. It is commonly assumed that this rate of energy
    expenditure is approximated by the rate of ATP production.}
}
\newglossaryentry{NETWORK ANALYSIS}{
    name={NETWORK ANALYSIS},
    description={An approach involving the analysis of gene networks. Gene networks are collections of
    functionally related genes (e.g. due to coexpression, protein-protein interactions, gene
    regulatory networks, etc.) where the topological relationships between the genes are
    known.}
}
\newglossaryentry{OVERWEIGHT AND OBESITY}{
    name={OVERWEIGHT AND OBESITY},
    description={These two terms have become well-defined entities with the widespread acceptance of
    the BMI as the metric of choice for classifying people for risk of disease resulting from
    excess weight. In people of European descent, overweight refers to a BMI in the range
    of 25 to 29.9 kg/m2 while obesity is defined as a BMI of 30 kg/m2 and more.}
}
\newglossaryentry{PATHWAY ANALYSIS}{
    name={PATHWAY ANALYSIS},
    description={An approach where the unit of analysis is a gene-set, also referred to as a pathway. A
    pathway is a collection of genes that are related to one another by some functional
    parameter. For genome-wide association studies, the goal of pathway analysis is to
    identify gene-sets that have a statistically significant excess of polymorphisms
    compared to random gene collections.}
}
\newglossaryentry{QUANTILE-QUANTILE PLOT}{
    name={QUANTILE-QUANTILE PLOT},
    description={A scatterplot created by plotting two sets of quantiles against one another. In case of
    GWA studies, this type of plot if often used to compare quantiles of the experimentally
    observed SNP association p-values versus quantiles calculated from a theoretical
    (normal) distribution.}
}
\newglossaryentry{REGULATORY MARKS}{
    name={REGULATORY MARKS},
    description={Chromatin modifications in gene regulatory regions, primarily involving post-translational
    modifications of DNA-associated histones (acetylation, methylation, phosphorylation,
    and ubiquitination).}
}