% cd /disks/PROJECT/Mickael/COMMUNICATION/JAV2016/;
% pdflatex Abstract_JAV2016.tex;
% evince Abstract_JAV2016.pdf &

\documentclass[11pt, a4paper]{article}
\usepackage[T1]{fontenc}
\usepackage[utf8]{inputenc}
\usepackage[hmargin=2.5cm, vmargin=2.5cm]{geometry}
\usepackage{amsmath}
\usepackage{xcolor}
% \usepackage[square, authoryear]{natbib}

\usepackage[english]{babel}
\selectlanguage{english}

\setlength{\parskip}{\baselineskip}%

\begin{document}
\section*{Single Nucleotide Polymorphisms Associated With Fasting Blood Glucose Trajectory And Type 2 Diabetes Incidence: \linebreak A Joint \mbox{Modelling} Approach}

\par{\small{{
    Mickaël Canouil$^{1}$, Philippe Froguel$^{1,2}$, Ghislain Rocheleau$^{1}$\\
    $^{1}$Univ. Lille, CNRS, CHU Lille, Institut Pasteur de Lille, UMR 8199 - EGID, F-59000 Lille, France\\
    $^{2}$Department of Genomics of Common Disease, Imperial College London, London, United Kingdom
}}}

\par{
In observational cohorts, longitudinal data are collected with repeated measurements at predetermined time points for many biomarkers, along with other covariates measured at baseline.
In these cohorts, time until a certain event of interest occurs is commonly reported and very often, a relationship will be observed between a biomarker repeatedly measured over time and that event.
Joint models were designed to efficiently estimate statistical parameters involved in these two types of data by combining a mixed model for the longitudinal biomarker trajectory and a survival model for the event risk.
Two main approaches have been proposed to account for the link between the two models in the shared random effect approach,
a function of the random effects (typically the expected current value of the biomarker) is introduced as an explanatory variable in the survival model.
In the latent class approach, the population is assumed to be divided in several latent classes with class-specific trajectories for the biomarker associated with class-specific risk function for the event.
}

\par{
Using genotypes assayed with the Metabochip DNA arrays (Illumina) from 4,500 subjects recruited in the French cohort D.E.S.I.R. (Données Épidémiologiques sur le Syndrome d’Insulino-Résistance),
we analysed a set of SNPs, selected from published genome-wide association studies, known to be associated with fasting blood glucose (FPG) and/or type 2 diabetes (T2D).
In our study, we focused in the shared random effect approach, where the event of interest is T2D and the longitudinal biomarker is FPG.
As expected, our analysis confirms a robust association between FPG and incident T2D. Effect sizes for these SNPs are consistent with those reported in the literature.
Our results suggest that SNP rs7903146 in TCF7L2 gene is associated with an increased risk of incident T2D and an elevated FPG level over time.
SNP rs560887 in G6PC2 (known to be associated with FPG) shows a statistically significant effect on the FPG trajectory and an unexpected protective effect on incident T2D,
which could be attributed to the limited number of incident T2D cases in our cohort.
}

\par{
These findings need to be replicated in other cohorts and further investigation of these effects remains to be done.
To the best of our knowledge, joint models have never been applied into a genetic epidemiology context and could help identify novel loci sharing effects on both glycaemic traits and T2D.
}
\end{document}